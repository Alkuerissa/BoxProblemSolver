\documentclass{article}
\usepackage{polski}
\usepackage[utf8]{inputenc}

\title{Problem pudełek}
\author{Ahmed Abdelkarim, Aleksandra Hernik}
\begin{document}
\maketitle
\section{Definicja problemu}
Problem polega na wybraniu spośród podanego zbioru pudełek, z których dla każdego podana jest szerokość i wysokość, największego ciągu pudełek takiego, że w każdym pudełku mieści się następne (możliwe są obroty o $\frac{\pi}{2}$). Pudełko mieści się w innym pudełku wtedy i tylko wtedy, gdy jego wysokość i szerokość są ostro mniejsze. 
\section{Opis wejścia i wyjścia}
\subsection{Wejście}
Plik tekstowy, zawierający w każdej linii oddzielone spacjami wymiary jednego pudełka, na przykład:\\
\texttt{10 5\\8 7\\2 3}
\subsection{Wyjście}
Plik tekstowy, w pierwszej linii zawierający jedną liczbę -- długość ciągu. Następne linie są w takim samym formacie jak plik wejściowy, ale zawierają posortowane malejąco pudełka, które tworzą najdłuższy ciąg. Ponadto program wyświetla graficzne przedstawienie rozwiązania.
\section{Opis rozwiązania}
Problem można sprowadzić do znalezienia najdłuższej ścieżki w acyklicznym grafie skierowanym. Każde pudełko jest wierzchołkiem grafu, a istnieje krawędź skierowana od $v_i$ do $v_j$ wtedy i tylko wtedy, gdy pudełko reprezentowane przez $v_j$ mieści się w pudełku reprezentowanym przez $v_i$. Tak zdefiniowany graf jest acykliczny ze względu na przyjętą definicję mieszczenia się pudełek - relacja mieszczenia się jest przechodnia i antysymetryczna (ze względu na opieranie się na relacji mniejszości posiadającej te cechy). Tworzenie grafu można przedstawić następująco:
\begin{verbatim}
graf = graf o n wierzchołkach
dla i = 0..n
	dla j = i..n
		jeśli pudełka[i] mieści się w pudełka[j]
			dodaj do grafu krawędź j->i
		jeśli pudełka[j] mieści się w pudełka[i]
			dodaj do grafu krawędź i->j
\end{verbatim}

W skierowanym grafie acyklicznym problem znalezienia najdłuższej ścieżki może zostać rozwiązany przez posortowanie topologiczne wierzchołków grafu, a następnie w ustalonej przez nie kolejności wybieranie najdłuższej ścieżki do każdego wierzchołka. Można to przedstawić następującym pseudokodem:
\begin{verbatim}
graf = stwórzGraf(pudełka)
wierzchołki = sortujTopologicznie(graf)
ścieżki = [0] * n
dla każdego v z wierzchołki
	dla każdego v_in mającego krawędź wchodzącą do v
		nowaOdległość = ścieżki[v_in] + 1
		jeśli nowaOdległość > ścieżki[v]
			ścieżki[v] = nowaOdległość 
\end{verbatim}
Dla czytelności pominięte zostało zapamiętywanie samej ścieżki -- w każdym kroku należy zapamiętywać nie tylko długość, ale dodatkowo poprzedni wierzchołek. Wykorzystane tu sortowanie topologiczne opisuje poniższy pseudokod:
\begin{verbatim}
start = wierzchołki bez krawędzi wchodzących
kolejność = []
dopóki start niepuste
	v = start[0]
	usuń v ze start
	dodaj v do kolejność
	dla każdej e = krawędź wychodząca z v
		usuń krawędź e z grafu
		jeśli drugi koniec e nie ma krawędzi wchodzących, dodaj go do start
\end{verbatim}
\section{Analiza złożoności}
\subsection{Złożoność czasowa}
Algorytm składa się z następujących czynności:
\begin{enumerate}
\item Jeśli $n$ to liczba pudełek, to złożoność stworzenia grafu to $O(n^2)$ -- trzeba sprawdzić każdy wierzchołek z każdym.
\item Złożoność sortowania topologicznego to $O(n^2)$ -- iterujemy po $n$ wierzchołkach, bo każdy wierzchołek jest dodawany do zbioru start dokładnie jeden raz, a od razu po rozpatrzeniu go jest usuwany. Dla każdego wierzchołka przechodzimy po wszystkich jego krawędziach, których może być co najwyżej $n-1$ -- krawędź do wszystkich pozostałych wierzchołków.
\item Złożoność znalezienia najdłuższej ścieżki to $O(n^2)$ -- dla każdego wierzchołka jest rozpatrywana każda wchodząca do niego krawędź.
\end{enumerate}
Wykonanie opisanych powyżej czynności daje ostateczną złożoność $O(n^2)$.
\subsection{Złożoność pamięciowa}
Pamięć potrzebna na zapamiętanie grafu to $O(n^2)$ -- $n$ wierzchołków oraz $O(n^2)$ krawędzi. W sortowaniu topologicznym konieczne jest stworzenie kopii grafu (ponieważ usuwamy krawędzie), a ponadto w tym oraz następnym kroku potrzebna jest stała liczba pomocniczych tablic o długości $n$, co nie zwiększa ostatecznej złożoności pamięciowej algorytmu ponad $O(n^2)$.
\section{Analiza poprawności}
Fakt, że przekształcenie problemu pudełek na problem najdłuższej ścieżki w grafie został opisany wcześniej i nie wymaga dalszych wyjaśnień. Sortowanie topologiczne jest możliwe, ponieważ graf jest acykliczny i skierowany. Algorytm na sortowanie topologiczne to algorytm Kahna. Dowód poprawności został przedstawiony w jego artykule \cite{Kahn}. 

Dowód poprawności algorytmu znajdowania najdłuższej ścieżki w acyklicznym grafie skierowanym można przeprowadzić indukcyjnie.

\textbf{Baza indukcji:} najdłuższa ścieżka kończąca się w każdym wierzchołku bez krawędzi wchodzących wynosi 0 -- prawda, bo składa się tylko z tego wierzchołka.

\textbf{Krok indukcyjny:} jeśli dla każdego $u$ takiego, że $u$ poprzedza $v$ w kolejności sortowania topologicznego, najdłuższa ścieżka kończąca się w wierzchołku $u$ została wyznaczona poprawnie, to algorytm poprawnie wyznaczy najdłuższą ścieżkę kończącą się w wierzchołku $v$. 

\textbf{Dowód kroku indukcyjnego:} niech $v$ to obecnie rozpatrywany wierzchołek, a $U$ to zbiór wierzchołków, z których istnieje krawędź do $v$. Z definicji sortowania topologicznego każdy wierzchołek $u \in U$, z którego istnieje krawędź do $v$, poprzedza go. To oznacza, że wszystkie wierzchołki ze zbioru $U$, które mogą się znajdować na ścieżce przed $v$ zostały już odwiedzone (została wyznaczona poprawna wartość). W związku z tym, najdłuższa ścieżka kończąca się w wierzchołku $v$ może zostać utworzona przez wybranie ze zbioru $U$ takiego wierzchołka, że ścieżka kończąca się w nim jest najdłuższa, i dodanie do niej wierzchołka $v$.

Jeśli najdłuższe ścieżki kończące się w każdym wierzchołku zostały wyznaczone poprawnie, najdłuższą ścieżką w grafie jest najdłuższa spośród tych ścieżek.

\begin{thebibliography}{20}
\bibitem{Kahn}
Kahn, Arthur B. (1962), \textit{Topological sorting of large networks}, \textit{Communications of the ACM}, 5 (11): 558–562
\end{thebibliography}
\end{document}
